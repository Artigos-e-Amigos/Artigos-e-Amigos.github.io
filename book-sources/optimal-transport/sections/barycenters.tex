% !TEX root = ../CourseOT.tex



%%%%%%%%%%%%%%%%%%%%%%%%%%%%%%%%%%%%%%%%%%%%%%%%%%%%%%%%%%%%%%%%%%%%%%%%%%%
\section{Barycenters}

%%%%%%%%%%%%%%%%%%%%%%%%%%%%%%%%%%%%%%%%%%%%%%%%%%%%%%%%%%%%%%%%%%%%%%%%%%%
\subsection{Frechet Mean over the Wasserstein Space}

 
This barycenter problem~\eqref{eq-wass-discr} was originally introduced by~\cite{Carlier_wasserstein_barycenter} following earlier ideas of~\cite{carlierekelandmatching}. They proved in particular uniqueness of the barycenter for $c(x,y)=\norm{x-y}^2$ over $\X=\RR^d$, if one of the input measure has a density with respect to the Lebesgue measure (and more generally under the same hypothesis as the one guaranteeing the existence of a Monge map, see Remark~\ref{rem-exist-mongemap}).

Given a set of input measure $(\be_s)_s$ defined on some space $\X$, the barycenter problem becomes 
	\eql{\label{eq-barycenter-generic}
		\umin{\al \in \Mm_+^1(\X)} \sum_{s=1}^S \la_s \MK_{\c}(\al,\be_s).
	}
	In the case where $\X=\RR^d$ and $c(x,y)=\norm{x-y}^2$,~\cite{Carlier_wasserstein_barycenter} shows that if one of the input measures has a density, then this barycenter is unique. 
	%
	Problem~\eqref{eq-barycenter-generic} can be viewed as a generalization of the problem of computing barycenters of points $(x_s)_{s=1}^S \in \X^S$ to arbitrary measures. Indeed, if $\be_s=\de_{x_s}$ is a single Dirac mass, then a solution to~\eqref{eq-barycenter-generic} is $\de_{x^\star}$ where $x^\star$ is a Fr\'echet mean solving~\eqref{eq-frechet-means}.
	%
	Note that for $c(x,y)=\norm{x-y}^2$, the mean of the barycenter $\al^\star$ is necessarily the barycenter of the mean, \emph{i.e.} 
	\eq{
		\int_\Xx x \d\al^\star(x) =  \sum_s \la_s \int_\Xx x \d\al_s(x), 
	}
	and the support of $\al^\star$ is located in the convex hull of the supports of the $(\al_s)_s$.
	%
	The consistency of the approximation of the infinite dimensional optimization~\eqref{eq-barycenter-generic} when approximating the input distribution using discrete ones (and thus solving~\eqref{eq-wass-discr} in place) is studied in~\cite{Carlier-NumericsBarycenters}.
	%
	Let us also note that it is possible to re-cast~\eqref{eq-barycenter-generic} as a multi-marginal OT problem, see Remark~\ref{eq-multimarg-bary}.
	

\todo{Write me: existence, dual, Monge map}

  
  
%%%%%%%%%%%%%%%%%%%%%%%%%%%%%%%%%%%%%%%%%%%%%%%%%%%%%%%%%%%%%%%%%%%%%%%%%%%
\subsection{1-D Case}


\todo{Write me}


%%%%%%%%%%%%%%%%%%%%%%%%%%%%%%%%%%%%%%%%%%%%%%%%%%%%%%%%%%%%%%%%%%%%%%%%%%%
\subsection{Gaussians Case}

\todo{Write me}
 
 
 
%%%%%%%%%%%%%%%%%%%%%%%%%%%%%%%%%%%%%%%%%%%%%%%%%%%%%%%%%%%%%%%%%%%%%%%%%%%
\subsection{Discrete Barycenters}
   
Given input histogram $\{\b_s\}_{s=1}^S$, where $b_s \in \simplex_{n_s}$, and weights $\la \in \simplex_S$, a Wasserstein barycenter is computed by minimizing
\eql{\label{eq-wass-discr}
	\umin{\a \in \simplex_n} \sum_{s=1}^S \la_s \MKD_{\C_s}(\a,\b_s)
}
where the cost matrices $\C_s \in \RR^{n \times n_s}$ need to be specified. 
%
A typical setup is ``Eulerian'', so that all the barycenters are defined on the same grid, $n_s=n$, $\C_s=\C=\distD^p$ is set to be a distance matrix, so that one solves
\eq{
	\umin{\a \in \simplex_n} \sum_{s=1}^S \la_s \WassD_p^p(\a,\b_s). 
}


The barycenter problem for histograms~\eqref{eq-wass-discr} is in fact a linear program, since one can look for the $S$ couplings $(\P_s)_s$ between each input and the barycenter itself
\eq{
	\umin{\a \in \simplex_n, (\P_s \in \RR^{n \times n_s})_s } \enscond{
		\sum_{s=1}^S \la_s \dotp{\P_s}{\C_s}
	}{
		\foralls s, \P_s^\top \ones_{n_s}=\a, \P_s^\top \ones_{n} = \b_s
	}.
}
Although this problem is an LP, its scale forbids the use generic solvers for medium scale problems. One can therefore resort to using first order methods such as subgradient descent on the dual~\cite{Carlier-NumericsBarycenters}.

%%%%%%%%%%%%%%%%%%%%%%%%%%%%%%%%%%%%%%%%%%%%%%%%%%%%%%%%%%%%%%%%%%%%%%%%%%%
\subsection{Sinkhorn for barycenters}

\todo{Explain the key difference with the regularized OT problem: here there is no more a ``canonical'' reference measure $\al \otimes \be$ since the barycenter is unknown. }

One can use entropic smoothing and approximate the solution of~\eqref{eq-wass-discr} using 
\eql{\label{eq-entropic-bary}
	\umin{\a \in \simplex_n} \sum_{s=1}^S \la_s \MKD_{\C_s}^\epsilon(\a,\b_s)
}
for some $\epsilon>0$. 
%
This is a smooth convex minimization problem, which can be tackled using gradient descent~\cite{CuturiBarycenter}. An alternative is to use descent method (typically quasi-Newton) on the semi-dual~\cite{2016-Cuturi-siims}, which is useful to integrate additional regularizations on the barycenter (e.g. to impose some smoothness).
%
A simple but effective approach, as remarked in~\cite{2015-benamou-cisc} is to rewrite~\eqref{eq-entropic-bary} as a (weighted) KL projection problem
\eql{\label{eq-bary-entropy-couplings}
	\umin{ (\P_s)_s } \enscond{ \sum_{s} \la_s \KLD( \P_s|\K_s ) }{
		\foralls s, \transp{\P_s}\ones_m = \b_s, \:
		\P_1\ones_1 =  \ldots = \P_S\ones_S
	}
}
where we denoted $\K_s \eqdef e^{-\C_s/\epsilon}$. Here, the barycenter $\a$ is implicitly encoded in the row marginals of all the couplings $\P_s \in \RR^{n \times n_s}$ as $\a = \P_1\ones_1 =  \ldots = \P_S\ones_S$.
%
As detailed in~\cite{2015-benamou-cisc}, one can generalize Sinkhorn to this problem, which also corresponds to iterative projection. This can also be seen as a special case of the generalized Sinkhorn detailed in Section~\ref{sec-generalized}.
%
The optimal couplings $(\P_s)_s$ solving~\eqref{eq-bary-entropy-couplings} are computed in scaling form as 
\eql{\label{eq-bary-opt}
	\P_s=\diag(\uD_s)\K\diag(\vD_s), 
}
and the scalings are sequentially updated as
\begin{align}\label{eq-sinkhorn-bary}
	\foralls s \in \range{1,S}, \quad \itt{\vD}_s &\eqdef \frac{\b_s}{\transp{\K}_s \it{\uD}_s}, \\
	\foralls s \in \range{1,S}, \quad  \itt{\uD}_s &\eqdef \frac{\itt{\a}}{\K_s \itt{\vD}_s}, \label{eq-sinkhorn-bary-2}\\
		\qwhereq
		\itt{\a} &\eqdef \prod_s (  \K_s \itt{\vD}_s )^{\la_s}. \label{eq-sinkhorn-bary-3}
\end{align}
An alternative way to derive these iterations is to perform alternate minimization on the variables of a dual problem, which detailed in the following proposition.

\begin{prop}
	The optimal $(\uD_s,\vD_s)$ appearing in~\eqref{eq-bary-opt} can be written as $(\uD_s,\vD_s) = ( e^{\fD_s/\epsilon},e^{\gD_s/\epsilon} )$ where $( \fD_s,\gD_s )_s$ are the solutions of the following program (whose value matches the one of \eqref{eq-entropic-bary})
	\eql{\label{eq-dual-bary-entropy}
		\umax{ ( \fD_s,\gD_s )_s } \enscond{
		\sum_s \la_s \pa{
			\dotp{\gD_s}{\b_s} - \epsilon \dotp{\K_s e^{\gD_s/\epsilon}}{e^{\fD_s/\epsilon}}
		}
		}{
			\sum_s \la_s \fD_s = 0
		}.
	}
\end{prop}

\begin{proof}
	Introducing Lagrange multipliers in~\eqref{eq-bary-entropy-couplings} leads to
	\begin{align*}
		\umin{ (\P_s)_s, \a } 
		\umax{ ( \fD_s,\gD_s )_s }
		\sum_{s} \la_s \Big(
			\epsilon \KLD( \P_s|\K_s )
			+ 
			\dotp{\a - \P_s\ones_m}{ \fD_s } \\
		\qquad\qquad	+			
			\dotp{\b_s - \transp{\P_s}\ones_m}{ \gD_s }
		\Big).
	\end{align*}
	Strong duality holds, so that one can exchange the min and the max, and gets
		\begin{align*}
		&\umax{ ( \fD_s,\gD_s )_s }
			\sum_s \la_s \pa{
				 \dotp{\gD_s}{\b_s}
		+ 
		\umin{\P_s} \epsilon \KLD( \P_s|\K_s ) - \dotp{\P_s}{\fD_s \oplus \gD_s }
		} \\
	&\qquad\qquad	+ \umin{\a}
			\dotp{ \sum_{s} \la_s \fD_s }{\a}.
	\end{align*}
	The explicit minimization on $\a$ gives the constraint $\sum_{s} \la_s \fD_s=0$ together with 
	\begin{align*}
		\umax{ ( \fD_s,\gD_s )_s }
			\sum_s \la_s 
				 \dotp{\gD_s}{\b_s}
		-
		\epsilon \KLD^*\pa{ \frac{\fD_s \oplus \gD_s}{\epsilon}|\K_s }
	\end{align*}
	where $\KLD^*(\cdot|\K_s)$ is the Legendre transform~\eqref{eq-legendre} of the function $\KLD^*(\cdot|\K_s)$.
	%
	This Legendre transform reads
	\eql{\label{eq-legendre-kl}
		\KLD^*(\VectMode{U}|\K) = \sum_{i,j} \K_{i,j} (e^{\VectMode{U}_{i,j}}-1), 
	}
	which shows the desired formula. 
	%
	To show~\eqref{eq-legendre-kl}, since this function is separable, one needs to compute
	\eq{
		\foralls (u,k) \in \RR_+^2, \quad
		\KLD^*(u|k) \eqdef \umax{r} ur -   \pa{ r \log(r/k) - r + k  }
	}
	whose optimality condition reads $u=\log(r/k)$, i.e. $r = k e^{u}$, hence the result.
\end{proof}

Minimizing~\eqref{eq-dual-bary-entropy} with respect to each $\gD_s$, while keeping all the other variable fixed, is obtained in closed form by~\eqref{eq-sinkhorn-bary}. 
%
Minimizing~\eqref{eq-dual-bary-entropy} with respect to all the $(\fD_s)_s$ requires to solve for $\a$ using~\eqref{eq-sinkhorn-bary-3} and leads to the expression~\eqref{eq-sinkhorn-bary-2}.

Figures~\ref{fig-barycenters-images} and~\ref{fig-bary-shapes} show applications to 2-D and 3-D shapes interpolation. Figure~\ref{fig-barycenters-surfaces} shows a computation of barycenters on a surface, where the ground cost is the square of the geodesic distance. For this figure, the computations are performed using the geodesic in heat approximation detailed in Remark~\ref{rem-geod-heat}. We refer to~\cite{2015-solomon-siggraph} for more details and other applications to computer graphics and imaging sciences.
