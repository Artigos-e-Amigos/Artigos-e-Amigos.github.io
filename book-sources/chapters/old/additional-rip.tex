

We denote $\bar x_s \in \RR^N$ the best $s$-term approximation of $x$, obtained by only keeping the $s$ largest coefficients in magnitude from $x$ and setting the others to 0.

We consider a solution $x^\star$ of
\eq{
	\umin{\norm{A \tilde x - y} \leq \epsilon} \normu{\tilde x}. 	
}
This note recall the proof from~\cite{candes-cras} of the following theorem

\begin{thm}[\cite{candes-cras}]\label{thm-rip}
	If $\de_{2s} \leq \sqrt{2}-1$ then there exists $C_0, C_1$ such that
	\eq{
		\normu{x^\star-x_0} \leq \frac{C_0}{\sqrt{s}}\norm{\bar x_s-x_0} + C_1 \epsilon.
	}
\end{thm}

Putting together Theorems~\ref{thm-rip-random}�and~\ref{thm-rip} gives the following recovery guarantee.

\begin{thm}\label{thm-rip-final}
	If $A$ is a sub-Gaussian random matrix, then provided
	\eql{\label{eq-thm-rip-final}
		P \geq C \de^{-2} s \ln(eN/s)		
	}
	with probability $1 - 2e^{ -\de^2 \frac{P}{2C} }$ on the draw of $A$, one has that for every $s$-sparse signal, 
	\eq{
		\normu{x^\star-x_0} \leq \frac{C_0}{\sqrt{s}}\norm{x_s-x_0} + C_1 \epsilon.
	}
	In particular, if $\norm{w}=0$ and $x_0$ is $s$-sparse, setting $\epsilon=0$, one has $x^\star=x_0$.
\end{thm}

The remaining of this section is devoted to proving Theorem~\ref{thm-rip}.



%%
\paragraph{Notations.}

We denote $x$ in place of $x_0$ and $x_s$ in place of $\bar x_s$.
%
We denote in the following $h = x^\star - x$ and denote $T_0$ the largest $s$ coefficients of $x$ in magnitude (so that $x_s=x_{T_0}$), $T_1$ the $s$ largest coefficients of $h_{T_0^c}$, $T_2$ the following $s$ largest coefficients of  $h_{T_0^c}$ and so on. We denote $T = T_0 \cup T_1$ which is an index set of size $2s$.


\begin{lem}\label{lemma-1}
	One has
	\eq{
		\sum_{j \geq 2} \norm{h_{T_j}} \leq \frac{1}{\sqrt{s}} \normu{h_{T_0^c}}
	}
\end{lem}
\begin{proof}
	By the definition of $T_j$ for $j \geq 2$, one has, for all $j \geq 2$
	\eq{
		\foralls i \in T_{j-1}, \quad
		\normi{h_{T_j}} \leq h_i, 
	}
	and hence
	\eq{
		\normi{h_{T_j}} \leq \frac{1}{s}\normu{h_{T_{j-1}}}.
	}
	This proves that
	\eq{
		\norm{h_{T_j}} \leq \sqrt{s} \normi{h_{T_j}} \leq \frac{1}{\sqrt{s}}\normu{h_{T_{j-1}}}
	}
	and thus 
	\eq{
		\sum_{j \geq 2} \norm{h_{T_j}} \leq \frac{1}{\sqrt{s}} 
		\sum_{j \geq 1} \normu{h_{T_j}} = \frac{1}{\sqrt{s}}  \normu{h_{T_0^c}}.
	}
\end{proof}

%%
\begin{lem}\label{lemma-2}
	One has 
	\eq{
		\normu{h_{T_0^c}} \leq \normu{h_{T_0}} + 2 \normu{x_{T_0^c}}
	}	
\end{lem}
\begin{proof}
	One has
	\begin{align*}
		\normu{x}
		& \geq \norm{x+h}
			& \text{because $x^\star$ is a minimizer} \\
		& = \normu{(x+h)_{T_0}} + \normu{(x+h)_{T_0^c}}
		  	& \\ 
		& \geq \normu{x_{T_0}} - \normu{h_{T_0}}
		+ \normu{h_{T_0^c}} - \normu{x_{T_0^c}}
			&\text{using the triangular inequality.}
	\end{align*}
	Decomposing the left hand size $\normu{x}=\normu{x_{T_0}} + \normu{x_{T_0^c}}$, 
	one obtains the result.
\end{proof}

%%
\begin{lem}\label{lemma-3}
	If $z$ and $z'$ have disjoints supports and $\norm{z} \leq s$ and $\normz{z'} \leq s$, 
	\eq{
		|\dotp{A z}{A z'}| \leq \de_{2s} \norm{z}\norm{z'}.
	}	
\end{lem}
\begin{proof}
	Using the RIP \eqref{eq-rip} since $z \pm z'$ has support of size $2s$ and the fact that $\norm{z \pm z'}^2 = \norm{z}^2 + \norm{z'}^2$, one has
	\eq{
		(1-\de_{2s}) \pa{ \norm{z}^2 + \norm{z'}^2 } 
		\leq \norm{A z \pm A z'}^2 \leq
		(1+\de_{2s}) \pa{ \norm{z}^2 + \norm{z'}^2 } .
	}
	One thus has using the parallelogram equality
	\eq{
		|\dotp{A z}{A z'}| =
		\frac{1}{4}
		| \norm{A z + A z'}^2 - \norm{A z - A z'}^2 |^2
		\leq
		\de_{2s} \norm{z}\norm{z'}.
	}
\end{proof}

Theorem \ref{thm-rip} requires bounding $\norm{h}$. We bound separately $\norm{h_T}$ and $\norm{h_{T^c}}$.

%%
\paragraph{Part 1: bounding $\norm{h_T}$.}

One has
\begin{align*}
	\norm{h_{T^c}} 
			& = \norm{\sum_{j \geq 2} h_{T_j}} \leq \sum_{j \geq 2} \norm{ h_{T_j} }
				& \text{using the triangular inequality} \\
			&\leq \frac{1}{\sqrt{s}} \normu{h_{T_0^c}}
				& \text{using Lemma \ref{lemma-1}}\\
			&\leq \frac{1}{\sqrt{s}} \normu{h_{T_0}} + \frac{2}{\sqrt{s}}\normu{x_{T_0^c}}
				&  \text{using Lemma \ref{lemma-2}}\\
			&\leq \frac{1}{\sqrt{s}} \normu{h_{T_0}} + 2 e_0
				& \text{denoting } e_0 = \frac{1}{\sqrt{s}}\normu{x_{T_0^c}} \\
			& \leq  \norm{h_{T_0}} + 2 e_0
				& \text{ using Cauchy-Schwartz}\\
			& \leq  \norm{h_{T}} + 2 e_0 
				& \text{because } T_0 \subset T.
\end{align*}
The final bound reads
\eql{\label{eq-5}
	\norm{ h_{T^c} } \leq \norm{h_{T}} + 2 e_0.
}


%%
\paragraph{Part 2: bounding $\norm{h_{T^c}}$.}

One has 
\begin{align*}
	\norm{h_{T}}^2 
		&\leq \frac{1}{1-\de_{2s}} \norm{ \Phi h_T }^2 & \text{using the RIP~\eqref{eq-rip}} \\
		& = \frac{A - B}{1-\de_{2s}} & \text{using } \Phi h_T = \Phi h - \sum_{j \geq 2} \Phi h_{T_j},
\end{align*}
where we have introduced 
\eq{
		A = \dotp{\Phi h_T}{\Phi h}
		\qandq
		B = \dotp{\Phi h_T}{ \sum_{j \geq 2 } \Phi h_{T_j} }	.
}

One has  
\begin{align*}
	|A|  & \leq \norm{\Phi h_T} \norm{\Phi h} 
			& \text{using Cauchy-Schwartz} \\
		 & \leq \sqrt{1+\de_{2s}} \norm{h_T} \norm{\Phi h}
		 	& \text{using the RIP~\eqref{eq-rip}} \\
		& \leq  \sqrt{1+\de_{2s}} \norm{h_T} 2 \epsilon 
		 & \text{using }  \norm{\Phi h} \leq \norm{\Phi x - y} + \norm{\Phi x^\star - y} \leq 2 \epsilon 
\end{align*}
The final bound reads
\eql{\label{eq-2}
	|A| \leq 2 \epsilon  \sqrt{1+\de_{2s}} \norm{h_T} 
}

One has 
\begin{align*}
	|B| &\leq |\dotp{\Phi h_{T_0}}{\sum_{j \geq 2} \Phi h_{T_j}}| + 
			 |\dotp{\Phi h_{T_1}}{\sum_{j \geq 2} \Phi h_{T_j}}|
			& \text{using the triangular inequality} \\
		& \leq \sum_{j \geq 2} |\dotp{\Phi h_{T_0}}{\Phi h_{T_j}}| + |\dotp{\Phi h_{T_0}}{\Phi h_{T_j}}| 
			& \text{using the triangular inequality} \\
		& \leq \sum_{j \geq 2} \de_{2s} \norm{h_{T_0}} \norm{h_{T_j}} + \de_{2s} \norm{h_{T_1}} \norm{h_{T_j}}
			& \text{using Lemma \ref{lemma-3}} \\
		& =  \de_{2s} (\norm{h_{T_0}} + \norm{h_{T_1}}) \sum_{j \geq 2} \norm{h_{T_j}}
			& \\
		& \leq \de_{2s} \sqrt{2} \norm{h_{T}} \sum_{j \geq 2} \norm{h_{T_j}} 
			& \text{$T_0$ and $T_1$ are disjoints} \\
		& \leq \frac{\sqrt{2}�\de_{2s}}{\sqrt{s}} \norm{h_T} \normu{h_{T_0^c}}
			& \text{using Lemma \ref{lemma-1}}
\end{align*}
The final bound reads
\eql{\label{eq-3}
	|B| \leq \frac{\sqrt{2}�\de_{2s}}{\sqrt{s}} \norm{h_T} \normu{h_{T_0^c}}.
}

Putting together \eqref{eq-2} and \eqref{eq-3} one obtains
\eq{
	\norm{h_{T}}^2 \leq \frac{\norm{h_{T}}}{ 1-\de_{2s} } \pa{
		\sqrt{1+\de_{2s}} 2 \epsilon  + \frac{2}{\sqrt{s}} \de_{2s} \normu{ h_{T_0^c} }
	}
}
thus
\begin{align*}
	\norm{h_{T}} 
	&\leq \al \epsilon + \frac{\rho}{\sqrt{s}} \normu{h_{T_0^c}}
		&\text{denoting}
		\choice{
			\al = 2 \frac{\sqrt{1+\de_{2s}}}{1-\de_{2s}}, \\
			\rho = \frac{\sqrt{2} \de_{2s}}{1-\de_{2s}}.
		}\\
	& \leq \al \epsilon + \frac{\rho}{\sqrt{s}} \normu{h_{T_0}}
			+ \frac{2 \rho}{\sqrt{s}} \normu{x_{T_0^c}} 
		& \text{using Lemma \ref{lemma-2}} \\
	& \leq \al \epsilon + \rho \norm{h_{T_0}} + 2 \rho e_0 
		& \text{using Cauchy-Schwartz}		\\
	& \leq \al \epsilon + \rho \norm{h_T} + 2 \rho e_0
		& \text{because } T_0 \subset T.
\end{align*}

Note that since $\de_{2s}�< \sqrt{2}-1$, one has $\rho < 1$.
This implies 
\eql{\label{eq-4}
	\norm{h_{T}} \leq
	\frac{\al}{1-\rho} \epsilon + \frac{2\rho}{1-\rho} e_0.
}


%%
\paragraph{Conclusion.}

One has
\begin{align*}
	\norm{h} &\leq \norm{h_T} + \norm{h_{T^c}} 
		& \text{using the triangular inequality} \\
	&\leq 2 \norm{h_T} + 2e_0 
		& \text{using \eqref{eq-5}} \\
	&\leq \frac{2\al}{1-\rho} \epsilon + 2\frac{1+\rho}{1-\rho} e_0
		& \text{using \eqref{eq-4}}
\end{align*}
which proves the theorem.