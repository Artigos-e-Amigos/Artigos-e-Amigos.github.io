\begin{maplegroup}
\begin{flushleft}
$b$ désigne la base de calcul. Il faut que 
\mapleinline{inert}{2d}{n(b-1)^2 < m;}{%
$n(b - 1)^{2} < m$%
}. 
\end{flushleft}

\end{maplegroup}
\begin{maplegroup}
\begin{mapleinput}
\mapleinline{active}{1d}{b := floor( evalf(sqrt(m/n))+1 ):}{%
}
\end{mapleinput}

\end{maplegroup}
\begin{maplegroup}
\begin{flushleft}
Calcule le produit point à point :
\end{flushleft}

\end{maplegroup}
\begin{maplegroup}
\begin{mapleinput}
\mapleinline{active}{1d}{cw_mult := \pproc{}(a,b)
\quad [seq( a[i]*b[i], i=1..n )]:
\pend \pproc{}:}{%
}
\end{mapleinput}

\end{maplegroup}
\begin{maplegroup}
\begin{flushleft}
Transforme un entier en vecteur :
\end{flushleft}

\end{maplegroup}
\begin{maplegroup}
\begin{mapleinput}
\mapleinline{active}{1d}{number2vector := \pproc{}(x)
\quad \plocal N, res, i, r, q, xx:
\quad N := floor( log(x)/log(b) )+1;
\quad res := []: xx := x:
\quad \pfor i \pfrom 1 \pto N \pdo
\quad \quad xx := iquo(xx,b,'r'):
\quad \quad res := [op(res), r]:
\quad \pend{}:
\quad \preturn{}(res):
\pend \pproc{}:}{%
}
\end{mapleinput}

\end{maplegroup}
\begin{maplegroup}
\begin{flushleft}
Transforme un vecteur en entier :
\end{flushleft}

\end{maplegroup}
\begin{maplegroup}
\begin{mapleinput}
\mapleinline{active}{1d}{vector2number := \pproc{}(v)
\quad add(v[k]*b^(k-1), k=1..nops(v));
\pend \pproc{}:}{%
}
\end{mapleinput}

\end{maplegroup}
\begin{maplegroup}
\begin{flushleft}
Calcule le produit de convolution :
\end{flushleft}

\end{maplegroup}
\begin{maplegroup}
\begin{mapleinput}
\mapleinline{active}{1d}{convol := \pproc{}(f,g)
\quad xFFT( cw_mult(xFFT(f,1),xFFT(g,1)), -1):
\pend \pproc{}:}{%
}
\end{mapleinput}

\end{maplegroup}
\begin{maplegroup}
\begin{flushleft}
Calcule le produit de deux entiers représentés sous forme de vecteurs
de taille 
\mapleinline{inert}{2d}{n;}{%
$n$%
}.
Attention, les 
\mapleinline{inert}{2d}{n;}{%
$n$%
}/2 dernières entrées des vecteurs doivent être nulles.
\end{flushleft}

\end{maplegroup}
\begin{maplegroup}
\begin{mapleinput}
\mapleinline{active}{1d}{prod_entiers := \pproc{}(x,y)
\quad \plocal res, i:
\quad res := convol(x,y):
\quad \pfor i \pfrom 1 \pto n-1 \pdo
\quad \quad res[i] := irem(res[i],b,'q'):
\quad \quad res[i+1] := res[i+1]+q;
\quad \pend{}:
\quad \preturn{}(res):
\pend \pproc{}:}{%
}
\end{mapleinput}

\end{maplegroup}
\begin{maplegroup}
\begin{flushleft}
Un test :
\end{flushleft}

\end{maplegroup}
\begin{maplegroup}
\begin{mapleinput}
\mapleinline{active}{1d}{hasard := rand(0..b-1):
xx := [seq( hasard(), i=1..n/2 ), seq(0, i=1..n/2)]; 
yy := [seq( hasard(), i=1..n/2 ), seq(0, i=1..n/2)]; 
x := vector2number(xx): y := vector2number(yy);
zz := prod_entiers(xx,yy);
evalb( vector2number(zz) = x*y ); # il doit y avoir égalité ...}{%
}
\end{mapleinput}

\mapleresult
\begin{maplelatex}
\mapleinline{inert}{2d}{xx := [4, 0, 0, 3, 3, 1, 0, 4, 0, 0, 0, 0, 0, 0, 0, 0];}{%
\[
\mathit{xx} := [4, \,0, \,0, \,3, \,3, \,1, \,0, \,4, \,0, \,0, 
\,0, \,0, \,0, \,0, \,0, \,0]
\]
%
}
\end{maplelatex}

\begin{maplelatex}
\mapleinline{inert}{2d}{yy := [3, 0, 4, 1, 4, 2, 3, 0, 0, 0, 0, 0, 0, 0, 0, 0];}{%
\[
\mathit{yy} := [3, \,0, \,4, \,1, \,4, \,2, \,3, \,0, \,0, \,0, 
\,0, \,0, \,0, \,0, \,0, \,0]
\]
%
}
\end{maplelatex}

\begin{maplelatex}
\mapleinline{inert}{2d}{y := 55853;}{%
\[
y := 55853
\]
%
}
\end{maplelatex}

\begin{maplelatex}
\mapleinline{inert}{2d}{zz := [2, 2, 1, 1, 3, 3, 2, 2, 1, 0, 3, 3, 2, 4, 2, 0];}{%
\[
\mathit{zz} := [2, \,2, \,1, \,1, \,3, \,3, \,2, \,2, \,1, \,0, 
\,3, \,3, \,2, \,4, \,2, \,0]
\]
%
}
\end{maplelatex}

\begin{maplelatex}
\mapleinline{inert}{2d}{true;}{%
\[
\mathit{true}
\]
%
}
\end{maplelatex}
\end{maplegroup}