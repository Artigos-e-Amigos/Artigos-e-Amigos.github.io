\begin{maplegroup}
\begin{flushleft}
Applique une matrice a un polynome :
\end{flushleft}

\end{maplegroup}
\begin{maplegroup}
\begin{mapleinput}
\mapleinline{active}{1d}{action_matrice := \pproc{}(A,p)
\quad \plocal g,v,m;
\quad v := array([[x],[y]]);
\quad m := evalm(A&*v);
\quad g := subs (\{x=m[1,1],y=m[2,1]\},p);
\quad \preturn{}(g)
\pend \pproc{}:}{%
}
\end{mapleinput}

\end{maplegroup}
\begin{maplegroup}
\begin{flushleft}
Un exemple invariant :
\end{flushleft}

\end{maplegroup}
\begin{maplegroup}
\begin{mapleinput}
\mapleinline{active}{1d}{A:=1/sqrt(2)*matrix(2,2,[[1,1],[1,-1]]);
expand( action_matrice(A,x^2+y^2) );}{%
}
\end{mapleinput}

\mapleresult
\begin{maplelatex}
\mapleinline{inert}{2d}{A := 1/2*sqrt(2)*matrix([[1, 1], [1, -1]]);}{%
\[
A := {\displaystyle \frac {1}{2}} \,\sqrt{2}\, \left[ 
{\begin{array}{rr}
1 & 1 \\
1 & -1
\end{array}}
 \right] 
\]
%
}
\end{maplelatex}

\begin{maplelatex}
\mapleinline{inert}{2d}{x^2+y^2;}{%
\[
x^{2} + y^{2}
\]
%
}
\end{maplelatex}

\end{maplegroup}
\begin{maplegroup}
\begin{flushleft}
Calcule l'opérateur de Reynolds :
\end{flushleft}

\end{maplegroup}
\begin{maplegroup}
\begin{mapleinput}
\mapleinline{active}{1d}{operateur_reynolds := \pproc{}(G,p)
\quad \plocal i,r;  
\quad r := (1/nops(G))*sum('action_matrice(G[k],p)','k'=1..nops(G));
\quad \preturn{}(r)
\pend \pproc{}:}{%
}
\end{mapleinput}

\end{maplegroup}
\begin{maplegroup}
\begin{flushleft}
Calcule des générateurs de l'anneau des invariants :
\end{flushleft}

\end{maplegroup}
\begin{maplegroup}
\begin{mapleinput}
\mapleinline{active}{1d}{polynomes_invariants := \pproc{}(G)
\quad \plocal i,j,r;
\quad r := [];
\quad \pfor i \pfrom 1 \pto nops(G) \pdo
\quad \pfor j \pfrom 0 \pto i \pdo
\quad \quad r := [op(r),expand(operateur_reynolds(G,x^j*y^(i-j)))];
\quad \pend \pdo{}: \pend \pdo{}:
\pend \pproc{}:}{%
}
\end{mapleinput}

\end{maplegroup}
\begin{maplegroup}
\begin{flushleft}
Un exemple en rapport aux codes auto-duaux, pour le cas où 2 divise
les poids des mots du code :
\end{flushleft}

\end{maplegroup}
\begin{maplegroup}
\begin{mapleinput}
\mapleinline{active}{1d}{B:=matrix(2,2,[[-1,0],[0,-1]]):
G:=[A,-A,B,-B]:
polynomes_invariants( G, 4 );}{%
}
\end{mapleinput}

\mapleresult
\begin{maplelatex}
\mapleinline{inert}{2d}{[0, 0, 1/4*x^2-1/2*x*y+3/4*y^2, 1/4*x^2-1/4*y^2+1/2*x*y,
3/4*x^2+1/2*x*y+1/4*y^2, 0, 0, 0, 0,
1/8*x^4-1/2*x^3*y+3/4*x^2*y^2-1/2*x*y^3+5/8*y^4,
1/8*x^4-1/4*x^3*y+3/4*x*y^3-1/8*y^4, 1/8*x^4+1/4*x^2*y^2+1/8*y^4,
1/8*x^4+3/4*x^3*y-1/4*x*y^3-1/8*y^4,
5/8*x^4+1/2*x^3*y+3/4*x^2*y^2+1/2*x*y^3+1/8*y^4];}{%
\maplemultiline{
[0, \,0, \,{\displaystyle \frac {1}{4}} \,x^{2} - {\displaystyle 
\frac {1}{2}} \,x\,y + {\displaystyle \frac {3}{4}} \,y^{2}, \,
{\displaystyle \frac {1}{4}} \,x^{2} - {\displaystyle \frac {1}{4
}} \,y^{2} + {\displaystyle \frac {1}{2}} \,x\,y, \,
{\displaystyle \frac {3}{4}} \,x^{2} + {\displaystyle \frac {1}{2
}} \,x\,y + {\displaystyle \frac {1}{4}} \,y^{2}, \,0, \,0, \,0, 
\,0,  \\
{\displaystyle \frac {1}{8}} \,x^{4} - {\displaystyle \frac {1}{2
}} \,x^{3}\,y + {\displaystyle \frac {3}{4}} \,x^{2}\,y^{2} - 
{\displaystyle \frac {1}{2}} \,x\,y^{3} + {\displaystyle \frac {5
}{8}} \,y^{4}, \,{\displaystyle \frac {1}{8}} \,x^{4} - 
{\displaystyle \frac {1}{4}} \,x^{3}\,y + {\displaystyle \frac {3
}{4}} \,x\,y^{3} - {\displaystyle \frac {1}{8}} \,y^{4},  \\
{\displaystyle \frac {1}{8}} \,x^{4} + {\displaystyle \frac {1}{4
}} \,x^{2}\,y^{2} + {\displaystyle \frac {1}{8}} \,y^{4}, \,
{\displaystyle \frac {1}{8}} \,x^{4} + {\displaystyle \frac {3}{4
}} \,x^{3}\,y - {\displaystyle \frac {1}{4}} \,x\,y^{3} - 
{\displaystyle \frac {1}{8}} \,y^{4},  \\
{\displaystyle \frac {5}{8}} \,x^{4} + {\displaystyle \frac {1}{2
}} \,x^{3}\,y + {\displaystyle \frac {3}{4}} \,x^{2}\,y^{2} + 
{\displaystyle \frac {1}{2}} \,x\,y^{3} + {\displaystyle \frac {1
}{8}} \,y^{4}] }
%
}
\end{maplelatex}

\end{maplegroup}
