\begin{maplegroup}
\begin{flushleft}
Définition des paramètres de la transformée
\end{flushleft}

\end{maplegroup}
\begin{maplegroup}
\begin{mapleinput}
\mapleinline{active}{1d}{s := 4: n := 2^s: m := 2^(2^(s-1)) + 1:}{%
}
\end{mapleinput}

\end{maplegroup}
\begin{maplegroup}
\begin{flushleft}
Sous-procédure récursive :
\end{flushleft}

\end{maplegroup}
\begin{maplegroup}
\begin{mapleinput}
\mapleinline{active}{1d}{FFT_rec := \pproc{}(f, signe, zeta)
\quad \plocal nn, n1, s, t, r;
\quad nn := nops(f); n1 := nn/2; # taille du vecteur
\quad \pif nn=1 \pthen \preturn{}(f) \pend \pif; # fin de l'algorithme
\quad # construction des deux sous-vecteurs de taille n1
\quad s := [ seq(f[2*k+1], k=0..n1-1) ];
\quad t := [ seq(f[2*k], k=1..n1) ];
\quad # calcul des deux sous-FFT :
\quad s := FFT_rec(s, signe, zeta^2 mod m);
\quad t := FFT_rec(t, signe, zeta^2 mod m);
\quad # mixage des deux résultats
\quad a := seq( s[k]+zeta^(-signe*(k-1))*t[k] mod m, k=1..n1 );
\quad b := seq( s[k]-zeta^(-signe*(k-1))*t[k] mod m, k=1..n1 );
\quad r := [a,b];
\quad \preturn{}(r);
\pend \pproc{}:}{%
}
\end{mapleinput}

\end{maplegroup}
\begin{maplegroup}
\begin{flushleft}
Procédure principale  (attention, le nom FFT est protégé en \Maple{} ...)
:
\end{flushleft}

\end{maplegroup}
\begin{maplegroup}
\begin{mapleinput}
\mapleinline{active}{1d}{xFFT := \pproc{}(f, signe)
\quad \plocal r;
\quad r := FFT_rec(f,signe,2);
\quad \pif signe=-1 \pthen r := 1/n*r mod m;
\quad \pelse r; \pend \pif
\pend \pproc{}: }{%
}
\end{mapleinput}

\end{maplegroup}
\begin{maplegroup}
\begin{flushleft}
Un test :
\end{flushleft}

\end{maplegroup}
\begin{maplegroup}
\begin{mapleinput}
\mapleinline{active}{1d}{hasard := rand(0..m-1):
x := [seq( hasard(), i=1..n )];
y := xFFT(x,+1); 
evalb( x = xFFT(y,-1) ); # On retombe bien sur nos pattes.}{%
}
\end{mapleinput}

\mapleresult
\begin{maplelatex}
\mapleinline{inert}{2d}{x := [179, 220, 230, 218, 49, 253, 197, 218, 67, 177, 136, 127, 190,
106, 210, 255];}{%
\[
x := [179, \,220, \,230, \,218, \,49, \,253, \,197, \,218, \,67, 
\,177, \,136, \,127, \,190, \,106, \,210, \,255]
\]
%
}
\end{maplelatex}

\begin{maplelatex}
\mapleinline{inert}{2d}{y := [5, 250, 28, 179, 190, 157, 195, 216, 198, 11, 13, 43, 5, 59,
49, 238];}{%
\[
y := [5, \,250, \,28, \,179, \,190, \,157, \,195, \,216, \,198, 
\,11, \,13, \,43, \,5, \,59, \,49, \,238]
\]
%
}
\end{maplelatex}

\begin{maplelatex}
\mapleinline{inert}{2d}{true;}{%
\[
\mathit{true}
\]
%
}
\end{maplelatex}

\end{maplegroup}
