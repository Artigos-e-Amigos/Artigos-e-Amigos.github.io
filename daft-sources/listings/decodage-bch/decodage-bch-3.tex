\begin{maplegroup}
\begin{center}
\textbf{[...] Suite du script précédent}
\end{center}

\end{maplegroup}
\begin{maplegroup}
\begin{flushleft}
\textbf{{\large 1ère partie : }}Résolution des équations pour 
\mapleinline{inert}{2d}{i = n-t;}{%
$i=n - t$%
} ... 
\mapleinline{inert}{2d}{n-1;}{%
$n - 1$%
} pour trouver 
\mapleinline{inert}{2d}{sigma;}{%
$\sigma $%
}[1] ... 
\mapleinline{inert}{2d}{sigma;}{%
$\sigma $%
}[t]
\end{flushleft}

\end{maplegroup}
\begin{maplegroup}
\begin{flushleft}
Calcule de l'équation polynomiale à résoudre (attention, on note la 
\mapleinline{inert}{2d}{epsilon;}{%
$\varepsilon $%
} \textit{transformée de Fourier} de l'erreur) :
\end{flushleft}

\end{maplegroup}
\begin{maplegroup}
\begin{mapleinput}
\mapleinline{active}{1d}{eqn := (1+add(sigma[i]*Z^i,i=1..t))*
\quad \quad \quad (add(epsilon[n-i]*Z^i,i=1..n)):
eqn := rem(eqn,Z^n-1,Z,'q');  # l'équation est modulo Z^n-1}{%
}
\end{mapleinput}

\mapleresult
\begin{maplelatex}
\mapleinline{inert}{2d}{eqn :=
(sigma[1]*epsilon[2]+sigma[3]*epsilon[4]+epsilon[1]+sigma[2]*epsilon[3
])*Z^14+(sigma[3]*epsilon[5]+epsilon[2]+sigma[1]*epsilon[3]+sigma[2]*e
psilon[4])*Z^13+(epsilon[3]+sigma[2]*epsilon[5]+sigma[1]*epsilon[4]+si
gma[3]*epsilon[6])*Z^12+(sigma[3]*epsilon[7]+sigma[2]*epsilon[6]+sigma
[1]*epsilon[5]+epsilon[4])*Z^11+(sigma[1]*epsilon[6]+epsilon[5]+sigma[
3]*epsilon[8]+sigma[2]*epsilon[7])*Z^10+(sigma[1]*epsilon[7]+sigma[3]*
epsilon[9]+epsilon[6]+sigma[2]*epsilon[8])*Z^9+(sigma[3]*epsilon[10]+s
igma[2]*epsilon[9]+sigma[1]*epsilon[8]+epsilon[7])*Z^8+(sigma[1]*epsil
on[9]+epsilon[8]+sigma[2]*epsilon[10]+sigma[3]*epsilon[11])*Z^7+(epsil
on[9]+sigma[1]*epsilon[10]+sigma[2]*epsilon[11]+sigma[3]*epsilon[12])*
Z^6+(sigma[2]*epsilon[12]+epsilon[10]+sigma[1]*epsilon[11]+sigma[3]*ep
silon[13])*Z^5+(sigma[1]*epsilon[12]+epsilon[11]+sigma[3]*epsilon[14]+
sigma[2]*epsilon[13])*Z^4+(sigma[1]*epsilon[13]+sigma[2]*epsilon[14]+e
psilon[12]+sigma[3]*epsilon[0])*Z^3+(sigma[1]*epsilon[14]+epsilon[13]+
sigma[3]*epsilon[1]+sigma[2]*epsilon[0])*Z^2+(epsilon[14]+sigma[1]*eps
ilon[0]+sigma[3]*epsilon[2]+sigma[2]*epsilon[1])*Z+sigma[3]*epsilon[3]
+epsilon[0]+sigma[2]*epsilon[2]+sigma[1]*epsilon[1];}{%
\maplemultiline{
\mathit{eqn} := ({\sigma _{1}}\,{\varepsilon _{2}} + {\sigma _{3}
}\,{\varepsilon _{4}} + {\varepsilon _{1}} + {\sigma _{2}}\,{
\varepsilon _{3}})\,Z^{14} + ({\sigma _{3}}\,{\varepsilon _{5}}
 + {\varepsilon _{2}} + {\sigma _{1}}\,{\varepsilon _{3}} + {
\sigma _{2}}\,{\varepsilon _{4}})\,Z^{13} \\
\mbox{} + ({\varepsilon _{3}} + {\sigma _{2}}\,{\varepsilon _{5}}
 + {\sigma _{1}}\,{\varepsilon _{4}} + {\sigma _{3}}\,{
\varepsilon _{6}})\,Z^{12} + ({\sigma _{3}}\,{\varepsilon _{7}}
 + {\sigma _{2}}\,{\varepsilon _{6}} + {\sigma _{1}}\,{
\varepsilon _{5}} + {\varepsilon _{4}})\,Z^{11} \\
\mbox{} + ({\sigma _{1}}\,{\varepsilon _{6}} + {\varepsilon _{5}}
 + {\sigma _{3}}\,{\varepsilon _{8}} + {\sigma _{2}}\,{
\varepsilon _{7}})\,Z^{10} + ({\sigma _{1}}\,{\varepsilon _{7}}
 + {\sigma _{3}}\,{\varepsilon _{9}} + {\varepsilon _{6}} + {
\sigma _{2}}\,{\varepsilon _{8}})\,Z^{9} \\
\mbox{} + ({\sigma _{3}}\,{\varepsilon _{10}} + {\sigma _{2}}\,{
\varepsilon _{9}} + {\sigma _{1}}\,{\varepsilon _{8}} + {
\varepsilon _{7}})\,Z^{8} + ({\sigma _{1}}\,{\varepsilon _{9}} + 
{\varepsilon _{8}} + {\sigma _{2}}\,{\varepsilon _{10}} + {\sigma
 _{3}}\,{\varepsilon _{11}})\,Z^{7} \\
\mbox{} + ({\varepsilon _{9}} + {\sigma _{1}}\,{\varepsilon _{10}
} + {\sigma _{2}}\,{\varepsilon _{11}} + {\sigma _{3}}\,{
\varepsilon _{12}})\,Z^{6} + ({\sigma _{2}}\,{\varepsilon _{12}}
 + {\varepsilon _{10}} + {\sigma _{1}}\,{\varepsilon _{11}} + {
\sigma _{3}}\,{\varepsilon _{13}})\,Z^{5} \\
\mbox{} + ({\sigma _{1}}\,{\varepsilon _{12}} + {\varepsilon _{11
}} + {\sigma _{3}}\,{\varepsilon _{14}} + {\sigma _{2}}\,{
\varepsilon _{13}})\,Z^{4} + ({\sigma _{1}}\,{\varepsilon _{13}}
 + {\sigma _{2}}\,{\varepsilon _{14}} + {\varepsilon _{12}} + {
\sigma _{3}}\,{\varepsilon _{0}})\,Z^{3} \\
\mbox{} + ({\sigma _{1}}\,{\varepsilon _{14}} + {\varepsilon _{13
}} + {\sigma _{3}}\,{\varepsilon _{1}} + {\sigma _{2}}\,{
\varepsilon _{0}})\,Z^{2} + ({\varepsilon _{14}} + {\sigma _{1}}
\,{\varepsilon _{0}} + {\sigma _{3}}\,{\varepsilon _{2}} + {
\sigma _{2}}\,{\varepsilon _{1}})\,Z + {\sigma _{3}}\,{
\varepsilon _{3}} + {\varepsilon _{0}} \\
\mbox{} + {\sigma _{2}}\,{\varepsilon _{2}} + {\sigma _{1}}\,{
\varepsilon _{1}} }
%
}
\end{maplelatex}

\end{maplegroup}
\begin{maplegroup}
\begin{flushleft}
Calcule les équations à résoudre, liste les valeurs de 
\mapleinline{inert}{2d}{epsilon;}{%
$\varepsilon $%
} connues, pour 
\mapleinline{inert}{2d}{i = 1;}{%
$i=1$%
} ... 
\mapleinline{inert}{2d}{2*t;}{%
$2\,t$%
}, puis évalue les équations :
\end{flushleft}

\end{maplegroup}
\begin{maplegroup}
\begin{mapleinput}
\mapleinline{active}{1d}{list_eqn1 := \{seq( coeff(eqn,Z,i), i=n-t..n-1 )\}:
epsilon_connu := \{seq( epsilon[i] = Syndi(p_recu,i), i=1..2*t )\};
eqn_eval1 := eval(list_eqn1, epsilon_connu);}{%
}
\end{mapleinput}

\mapleresult
\begin{maplelatex}
\mapleinline{inert}{2d}{epsilon_connu := \{epsilon[2] = alpha^3+alpha^2, epsilon[1] =
alpha^3, epsilon[6] = alpha^3+1, epsilon[5] = 1, epsilon[4] =
alpha^3+alpha^2+alpha+1, epsilon[3] = alpha^3+alpha+1\};}{%
\maplemultiline{
\mathit{epsilon\_connu} :=  \\
\{{\varepsilon _{2}}=\alpha ^{3} + \alpha ^{2}, \,{\varepsilon _{
1}}=\alpha ^{3}, \,{\varepsilon _{6}}=\alpha ^{3} + 1, \,{
\varepsilon _{5}}=1, \,{\varepsilon _{4}}=\alpha ^{3} + \alpha ^{
2} + \alpha  + 1, \,{\varepsilon _{3}}=\alpha ^{3} + \alpha  + 1
\} }
%
}
\end{maplelatex}

\begin{maplelatex}
\mapleinline{inert}{2d}{eqn_eval1 :=
\{sigma[1]*(alpha^3+alpha^2+alpha+1)+alpha^3+alpha+1+sigma[3]*(alpha^3
+1)+sigma[2],
sigma[2]*(alpha^3+alpha^2+alpha+1)+alpha^3+alpha^2+sigma[3]+sigma[1]*(
alpha^3+alpha+1),
alpha^3+sigma[3]*(alpha^3+alpha^2+alpha+1)+sigma[2]*(alpha^3+alpha+1)+
sigma[1]*(alpha^3+alpha^2)\};}{%
\maplemultiline{
\mathit{eqn\_eval1} := \{{\sigma _{1}}\,(\alpha ^{3} + \alpha ^{2
} + \alpha  + 1) + \alpha ^{3} + \alpha  + 1 + {\sigma _{3}}\,(
\alpha ^{3} + 1) + {\sigma _{2}},  \\
{\sigma _{2}}\,(\alpha ^{3} + \alpha ^{2} + \alpha  + 1) + \alpha
 ^{3} + \alpha ^{2} + {\sigma _{3}} + {\sigma _{1}}\,(\alpha ^{3}
 + \alpha  + 1),  \\
\alpha ^{3} + {\sigma _{3}}\,(\alpha ^{3} + \alpha ^{2} + \alpha 
 + 1) + {\sigma _{2}}\,(\alpha ^{3} + \alpha  + 1) + {\sigma _{1}
}\,(\alpha ^{3} + \alpha ^{2})\} }
%
}
\end{maplelatex}

\end{maplegroup}
\begin{maplegroup}
\begin{flushleft}
Met sous forme matricielle les équations :
\end{flushleft}

\end{maplegroup}
\begin{maplegroup}
\begin{mapleinput}
\mapleinline{active}{1d}{m1 := matrix(t,t):
b1 := vector(t):
i := 1:
\pfor eq \pin eqn_eval1 \pdo
\quad \pfor j \pfrom 1 \pto t \pdo
\quad \quad m1[i,j] := coeff(eq,sigma[j],1);
\quad \pend \pdo:
\quad b1[i] := eval( eq, [seq(sigma[k]=0,k=1..t)] );
\quad i := i+1:
\pend \pdo:}{%
}
\end{mapleinput}

\end{maplegroup}
\begin{maplegroup}
\begin{flushleft}
Calcule les valeurs de 
\mapleinline{inert}{2d}{sigma;}{%
$\sigma $%
} en résolvant le système :
\end{flushleft}

\end{maplegroup}
\begin{maplegroup}
\begin{mapleinput}
\mapleinline{active}{1d}{sigma_val := Linsolve(m1,b1) mod 2:
sigma_connu := \{ seq(sigma[i]=sigma_val[i], i = 1..t) \};}{%
}
\end{mapleinput}

\mapleresult
\begin{maplelatex}
\mapleinline{inert}{2d}{sigma_connu := \{sigma[1] = alpha^3+1, sigma[2] =
alpha^3+alpha^2+alpha, sigma[3] = alpha^2+1\};}{%
\[
\mathit{sigma\_connu} := \{{\sigma _{1}}=\alpha ^{3} + 1, \,{
\sigma _{2}}=\alpha ^{3} + \alpha ^{2} + \alpha , \,{\sigma _{3}}
=\alpha ^{2} + 1\}
\]
%
}
\end{maplelatex}

\end{maplegroup}